
% DELETED!
% DELETED!
% DELETED!
% DELETED!
% DELETED!

% DELETED!
\newcommand{\sufstats}[1]{\bm{g}\left(#1\right)}
\renewcommand{\exp}[1]{\mbox{exp}\left\{#1\right\}}
\renewcommand{\log}[1]{\mbox{log}\left\{#1\right\}}
\newcommand{\transpose}[1]{{#1}^{\bm{t}}} 
\renewcommand{\t}[1]{\transpose{#1}}

\newcommand{\s}[1]{\sufstats{#1}}
\newcommand{\SUFF}{\bm{S}}
\newcommand{\Suff}{\bm{G}}
\newcommand{\suff}{\bm{g}}

\newcommand{\coef}{\bm{\theta}}
\newcommand{\weight}{\bm{w}}
\newcommand{\Weight}{\bm{W}}

% Objects
% DELETED!
% DELETED!
\newcommand{\Graph}{\bm{Y}}
\newcommand{\graph}{\bm{y}}
\newcommand{\g}{\graph}
\newcommand{\GRAPH}{\mathcal{Y}}
\newcommand{\Adjmat}{A}
\newcommand{\adjmat}{a}
\newcommand{\ADJMAT}{\bm{A}}

\newcommand{\INDEPVAR}{\bm{X}}
\newcommand{\Indepvar}{X}
\newcommand{\indepvar}{x}

\newcommand{\normconst}{\kappa_{\GRAPH}\left(\coef\right)}

\graphicspath{{./figures/}{.}{./terms/}}


%% NEED THIS FOR CANCY TEX
\usepackage{pstricks}

% Colors
\definecolor{USCCardinal}{HTML}{990000} % 153 0 0 in RGB
\definecolor{USCGold}{HTML}{FFCC00}
\definecolor{USCGray}{HTML}{CCCCCC}

% \bibliography{bibliography.bib}

\def\ergmito{ERGM\textit{ito}}
\def\ergmitos{\ergmito{}\textit{s}}
% Mathematical functions
\newcommand{\isone}[1]{{\boldsymbol{1}\left( #1 \right)}}
\newcommand{\f}[1]{{f\left(#1\right) }}

% Define a command with an optional argument

% \usepackage[]{xparse}
% \NewDocumentCommand{\Prcond}{m m o}{
%   \IfNoValueTF{#3}{%
%   {\mbox{Pr}\left(#1;\;#2\right)} %
%   }{%
%   {\mbox{Pr}_{#3}\left(#1;\;#2\right)} %
%   }
% }

\renewcommand{\Pr}[1]{{\mbox{Pr}_{\GRAPH,\coef}\left(#1\right) }}
\newcommand{\Prcond}[2]{%
  {\mbox{Pr}_{\GRAPH,\coef}\left(#1\left|\;#2\right.\right)}%
  }
\newcommand{\fcond}[2]{{f\left(#1|#2\right) }}
\newcommand{\Expected}[1]{{\mathbb{E}\left\{#1\right\}}}
\newcommand{\ExpectedCond}[2]{{\mathbb{E}\left\{#1\vphantom{#2}\right|\left.\vphantom{#1}#2\right\}}}
\renewcommand{\exp}[1]{\mbox{exp}\left\{#1\right\}}

\newcommand{\Likelihood}[2]{\text{L}\left(#1 \left|\vphantom{#1}#2\right.\right)}

\newcommand{\loglik}[1]{l\left(#1\right)}